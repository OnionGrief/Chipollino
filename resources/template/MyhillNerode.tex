%include "TMtheory.tex"
\section{MyhillNerode}
%begin detailed
\begin{frame}{Эквивалентность слов в DFA}
    Пусть дан DFA $A$. Положим $w_1 {\equiv_A} w_2 \iff \exists q_i({q_0}\overset{w_1}{\rar}{q_i} \& {q_0}\overset{w_2}{\rar}{q_i})$.
    Если $w_1 {\equiv_A} w_2$, тогда $\forall z({w_1}z \in \Lang(A) \iff  {w_2}z \in \Lang(A))$

    $w_1 {\equiv_\Lang} w_2 \iff \forall z({w_1}z \in \Lang \iff  {w_2}z \in \Lang)$. Это отношение разбивает $\Lang$ на классы эквивалентности.
    \begin{block}{\bf Теорема Майхилла-Нероуда}
    Язык $\Lang$ регулярен тогда и только тогда, когда множество его классов эквивалентности по ${\equiv_\Lang}$ конечно.   
    \end{block}
\end{frame}
%end detailed
\begin{frame}{$\MyhillNerode\TypeIs\NFATYPE\to\IntTYPE$}
	Автомат:

	%template_oldautomaton

	Таблица классов эквивалентности:

	%template_table

	Итоговое число классов эквивалентности:
	%template_result

\end{frame}

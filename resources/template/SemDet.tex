\section{Семантический детерминизм}
%begin detailed
\begin{frame}{Семантический детерминизм}
    \vspace{-5pt}
     Язык состояния $q$ — это $\{w | q \xrightarrow{\text{$w$}} q\}$, где $q_f$  - финальное состояние. НКА $A$ семантически детеминирован, если для всякой неоднозначности $q_i \xrightarrow{\text{$a$}} {q_{j_1}} , ..., {q_i \xrightarrow{\text{$a$}} q_{j_k}}$ существует такое состояние $q_{j_s}$, что языки всех  $q_{j_t}$ $(1 \leqslant t \leqslant k)$ вкладываются в его язык (это означает, что переход $q_i \xrightarrow{\text{$a$}} {q_{j_s}}$ надёжен: если слово распознаётся автоматом, оно обязательно будет распознано после такого перехода).
     Языки состояний при этом строятся с помощью производных: в качестве аргумента производной достаточно взять произвольный префикс $v$, соответствующий переходу $q_0 \xrightarrow{\text{$v$}} q$.
\end{frame}
\begin{frame}{Алгоритм проверки семантического детерминизма}
    \vspace{-5pt}
    \begin{itemize}
        \item Разметка автомата
        \item Получение языков состояний, с помощью производных
        \item Поиск неоднозначностей
        \item Поиск состояния с надёжным переходом для каждой неоднозначности
    \end{itemize}
    
\end{frame}
%end detailed
\begin{frame}{Предикат $\SemDet\TypeIs\NFATYPE\to\BooleanTYPE$}
    \vspace{-5pt}
	oldautomaton:

	%template_oldautomaton

    Неоднозначные переходы и надёжные переходы:

	%template_ambiguous transitions


    Семантический детерминизм:

    %template_semdet result    
\end{frame}
%begin detailed
% descriptive documentation : section
\section{Основные понятия}
\begin{frame}{Частичные производные}
    \vspace{-5pt}
    $\alpha_{c}(R)$ — это регулярное выражение $R'$ такое, что если $w \in \Lang(R')$, то $cw \in \Lang(R)$. Обратное не обязательно выполняется. Вычислить частичные производные можно по следующему рекурсивному алгоритму.
    \[\alpha_{c}(c) = \{\empt\}\] % the Derivative placeholder 1 % the Derivative placeholder displaystyle
    \[\alpha_{c}(c') = \varnothing\] % the Derivative placeholder 2 % the Derivative placeholder displaystyle
    \[\alpha_{c}(\empt) = \varnothing\] % the Derivative placeholder 3 % the Derivative placeholder displaystyle
    \begin{equation*}
        \alpha_{c}(r_{1} r_{2}) =
        \begin{cases}
            \{r r_{2} | r \in \alpha_{c}(r_{1})\} \cup \alpha_{c}(r_{2}) & \text {если} \empt \in \Lang(r_{1}) \\ % the Derivative placeholder 4 % the Derivative placeholder displaystyle
            \{r r_{2} | r \in \alpha_{c}(r_{1})\}                        & \text {иначе}                       % the Derivative placeholder 5 % the Derivative placeholder displaystyle
        \end{cases}
    \end{equation*}
    \[\alpha_{c}(\bot) = \varnothing\] % the Derivative placeholder 6 % the Derivative placeholder displaystyle
    \[\alpha_{c}(r_{1}|r_{2}) = \alpha_{c}(r_{1}) \cup \alpha_{c}(r_{2})\] % the Derivative placeholder 7 % the Derivative placeholder displaystyle
    \[\alpha_{c}(r\star) = \{r'r\star | r' \in \alpha_{c}(r)\}\] % the Derivative placeholder 8 % the Derivative placeholder displaystyle
    Автомат Антимирова аналогичен автомату Брзозовски, но состояния представляют собой элементы $\alpha_{w}$, а не $\delta_{w}$. Упрощать по ACI состояния не требуется — их множество и так конечно.
\end{frame} % descriptive documentation
%end detailed

\section{Автомат Антимирова}
\begin{frame}{Пример автомата Антимирова}
    \vspace{-5pt}
    %template_initial_regex % the initial regexp placeholder

    Тогда (производные, равные пустому множеству, здесь опущены):\\
    %template_derivative % the Derivative placeholder 1
    Автомат Антимирова:

    %template_antimirov % the Antimirov diagram placeholder
\end{frame}

%begin detailed
% overall documentation : section 
\section{Обсуждение}
\begin{frame}{Дополнительные сведения}
    \begin{block}{\bf Связь с автоматом Томпсона}
        Автомат Антимирова может быть получен из автомата Томпсона путем последовательного применения к нему следующих операций:
        \[\RemEps(\DeAnnote(\Minimize(\RemEps(\Annote(\Thompson(r))))))\] % the formula Antimirov placeholder displaystyle
    \end{block}
    \begin{block}{\bf Теорема}
        Пусть $r$ -- взвешенное регулярное выражение над $K$. Если $K$ является $null-k$-замкнутым для $r$, то $\Antimirov(r)$ может быть вычислен за $O(m log m + mn)$ путем применения удаления $\empt$-переходов и минимизации.
    \end{block}
    Мы будем говорить, что $K$ является $null-k$-замкнутым для $\alpha$, если $\exists k \geq 0$, такое, что для каждого подтерма $\beta^*$ $null(\beta)^* = \oplus_{i=0}^{k} null(\beta)^i$.
\end{frame}
%end detailed
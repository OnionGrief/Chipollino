%begin detailed
\section{Длина накачки}
\begin{frame}{Лемма о накачке}
    \vspace{-5pt}
    Если $G$ — это регулярная грамматика, то существует $n \in \mathbb{N}$, что $\forall w(w \in \Lang(G) \:  \& \: |w| > n \Rightarrow \exists w_1, w_2, w_3(|w_2| > 0 \: \&  \: |w_1| + |w_3| \leqslant n \: \& \: w = w_1w_2w_3 \: \& \: \forall k(k\geqslant0 \Rightarrow w_1w_2^kw_3 \in \Lang(G))$. Длина накачки $G$ - минимальное из таких $n$.
    \begin{block}{\bf Алгоритм поиска длины накачки}
    Пусть $R\in\RegExp$. По возрастанию значения $n$:
    \begin{itemize}
        \item Рассмотреть в $R$ все возможные префиксы $w$ длины $n$ и по каждому из них построить производную Брзозовски.
        \item Перебрать инфиксы в пределах $n$-префиксов на предмет возможности накачки: а именно, если префикс $w$ в выражении $w\delta_w(R)$ допускает разбиение на $w_1w_2w_3$, то он накачивается $\Leftrightarrow \: \Lang((w_1(w_2) \star w_3\delta_w(R)) \subseteq \Lang(R)$.
    \end{itemize}
    Если все $n$-префиксы накачиваются, то $n$ и есть искомая длина накачки.
    \end{block}
\end{frame} % descriptive documentation
%end detailed
\begin{frame}{Вычисление $\PumpLength\TypeIs\RegexTYPE\to\IntTYPE$}
    \vspace{-5pt}
    Регулярное выражение:
	%template_oldregex

	Длина накачки:
    %template_pumplength1 % founded pump len

    Накачиваемые префиксы:
    %template_pumplength2 % prefixes
    
	%template_cach

\end{frame}


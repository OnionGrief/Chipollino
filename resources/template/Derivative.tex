\section{Производные регулярных выражений}
%begin detailed
\begin{frame}{Производные Бзрозовски}
    \vspace{-5pt}
    \begin{block}{\bf Определение}
    Множество $a^{-1} U = \{w | aw \in U\}$ называется производным Бзрозовски множества $U$ относительно $a$. Если $\epsilon \in a^{-1} U$, тогда a распознаётся выражением U.
    \end{block}
    \begin{itemize}
        \item $a^{-1}(\varnothing) = a^{-1}(\epsilon) = \varnothing$
        \item $a^{-1}(b) = \varnothing, if\:b \neq a$
        \item $a^{-1}(b) = \epsilon, if\ :b = a$
        \item $a^{-1}(E|F) = a^{-1}(E) \union a^{-1}(F)$
        \item $a^{-1}(EF) = a^{-1}(E)F \union a^{-1}(F), if \: \epsilon \in \Lang(E)$
        \item $a^{-1}(EF) = a^{-1}(E)F, if \: \epsilon \notin \Lang(E)$
        \item $a^{-1}(E \star ) = a^{-1}(E) E \star$
    \end{itemize}
\end{frame}
\begin{frame}{Частичные производные}
    \begin{block}{\bf Определение}
    Для регулярного выражения $R$ частичная производная $D_a(R)$ — это регулярное выражение $R'$ такое, что если $w \in \Lang(R')$, то $aw \in L(R)$. Обратное не обязательно выполняется.
    \end{block}
    Частичная производная регулярного выражения по символу $a$ определяется следующими правилами:
    \begin{itemize}
        \item $D_a(\varnothing) = D_a(\epsilon) = \varnothing$
        \item $D_\epsilon(a) = a$
        \item $D_a(b) = \varnothing, if\:b \neq a$
        \item $D_a(b) = \epsilon, if\:b = a$
        \item $D_a(E | F) = D_a(E) | D_a(F)$
        \item $D_a(EF) = D_a(E)F | D_a(F), if \: \epsilon \in \Lang(E)$
        \item $D_a(EF) = D_a(E)F, if \: \epsilon \notin \Lang(E)$
        \item $D_a(E \star ) = D_a(E) E \star$
        \item $D_\epsilon(E) = E$
    \end{itemize}
\end{frame} % descriptive documentation
%end detailed